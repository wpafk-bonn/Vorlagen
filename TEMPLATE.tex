\documentclass[a4paper]{scrartcl}
\usepackage[ngerman]{babel}
\usepackage[utf8]{inputenc}
\usepackage{graphicx}
%\usepackage{array}
\usepackage{enumitem}
%\usepackage{pdfpages}
%\usepackage{listings}
\usepackage[usenames,dvipsnames,svgnames,table]{xcolor}
%\usepackage{amsmath}
\usepackage{amssymb}
\usepackage[T1]{fontenc}
\usepackage[ttdefault=true]{AnonymousPro}

\renewcommand*\familydefault{\ttdefault} %% Only if the base font of the document is to be typewriter style

\usepackage{geometry}
\geometry{margin=2cm}

\setlength{\parskip}{0.5em}
\setlength{\parindent}{0em}

\newcommand{\fullcheck}{\raisebox{-.8\dp\strutbox}{\includegraphics[width=14pt]{Check.pdf}}}
\newcommand{\semicheck}{\raisebox{-.8\dp\strutbox}{\includegraphics[width=14pt]{Semicheck.pdf}}}
\newcommand{\nocheck}{\raisebox{-.8\dp\strutbox}{\includegraphics[width=14pt]{Nocheck.pdf}}}
\newcommand{\dontknow}{\raisebox{-.8\dp\strutbox}{\includegraphics[width=14pt]{Dontknow.pdf}}}
\newcommand{\notrev}{\raisebox{-.8\dp\strutbox}{\includegraphics[width=14pt]{Notrev.pdf}}}
\newcommand{\sym}[1]{
\ifcase#1 \item[$\Box$]
\or \item[\fullcheck]
\or \item[\semicheck]
\or \item[\nocheck]
\or \item[\dontknow]
\or \item[\notrev]
\else \item[$\Box$]
\fi}

\usepackage[colorlinks=true]{hyperref}


%=============================VARIABLEN====================================
\newcommand{\fachschaft}{FACHSCHAFT}
\newcommand{\gremium}{FSV} %Bei der Wahl gewähltes Gremium.
\newcommand{\wahltermin}{DD. -- DD. MMMM YYYY}
\newcommand{\sitzungstermin}{DD. MMMM YYYY}
\newcommand{\vorsitz}{Fiona Oberem}
\newcommand{\anwesende}{\vorsitz\ (Vorsitz), Christoph Heinen, Ilka Fisser, Mara Weber, Fabienne Hering}
\newcommand{\berichterstatter}{NN.} %Bevollmächtigte Vertretung der geprüften Fachschaft.
%==========================================================================


\title{Wahlprüfbericht}
\subtitle{\gremium-Wahl \fachschaft, \wahltermin}
\author{Wahlprüfungsausschuss der Fachschaftenkonferenz (WPAF)}
\date{\sitzungstermin}

\begin{document}

\maketitle

Der Wahlprüfungsausschuss der Fachschaftenkonferenz (WPAF) hat am \sitzungstermin \ die Fachschaftswahl der Fachschaft \fachschaft\ geprüft. \\
Anwesend waren: \anwesende.\\ 
Berichterstatter*in: \berichterstatter

\textbf{Legende:}

\begin{tabular}{|c|l|}\hline
\fullcheck & In Ordnung \\\hline
\semicheck & Teilweise / kleinere Mängel \\\hline
\nocheck & Fehlt / Fehlerhaft\\\hline
\dontknow & Unbekannt / Unklar\\\hline
\notrev & Nicht zutreffend / Nicht relevant\\\hline
\end{tabular}


\section{Dokumente und Unterlagen}

Die folgenden Dokumente und Unterlagen wurden zugesandt:
%=============================================================
%== Befehl /sym{n} erstellt die in der Legende angegebenen Symbole. ========
%== n=1: In Ordnung, n=2: Teilweise / kleinere Mängel, n=3: Fehlt / Fehlerhaft ==
%== n=4: Unbekannt / Unklar, n=5: Nicht zutreffend / Nicht relevant =========
%=============================================================

\begin{itemize}[label=$\Box$]
\sym{0} Wahlbekanntmachung (Kopie)
\sym{0} Protokolle und Sitzungseinladungen: (Kopien)

\begin{itemize}[label=$\Box$]
\sym{0} Wahl von Wahlleitung und Wahlausschuss
\sym{0} Festlegung des Wahltermins
\sym{0} Wahlausschusssitzungen % mindestens drei Sitzungen §§13.4 ff
    \begin{itemize}[label=$\Box$]
    \sym{0} Konstituierende Sitzung des Wahlausschusses
    \sym{0} Beschlüsse nach §13 Abs. 5 FSWO
    \sym{0} Prüfung / Zulassung der eingereichten Kandidaturen, Briefwahlanträge und Einsprüche gegen das Wählendenverzeichnis 
    \sym{0} Zusätzliche Sitzung im Falle einer Nachfrist
    \sym{0} Sitzung der Auszählung der Wahl
    \end{itemize}
\sym{0} Protokoll der Wahlvollversammlung % wenn gegeben
\sym{0} konstituierende Sitzung nach der Wahl
\end{itemize}

\sym{0} Anträge zum Wahlverfahren (Originale)
\sym{0} Mitgliederliste von FSV und FSR zum Zeitpunkt der Wahl des Wahlausschusses
\sym{0} Liste der an der Auszählung beteiligten Wahlhelfenden
\sym{0} Wahlergebnis (Kopie)
\sym{0} Bekanntmachung der Wahlvorschläge (Kopie)
\sym{0} Urnenbuch (Original)
\sym{0} Stimmzettel (Originale)
\sym{0} Wählendenverzeichnis (Original)
\sym{0} Wahlvorschläge und Kandidaturen (Originale, ALLE, auch abgelehnte)
\sym{0} Briefwahlanträge (Originale)
\end{itemize}

% Das Protokoll einer Wahlausschusssitzung mit Zulassung der Wahlvorschläge und Festlegung des Wahlverfahrens im Sonderfall ist nicht vorhanden.

% Die FSR- und FSV-Zusammensetzung wurde nicht / teilweise zugesandt, konnte jedoch alternativ ermittelt werden. 

% Briefwahlanträge wurden nicht übermittelt, daher ist davon auszugehen, dass keine vorlagen.

% Wahlvorschläge wurden angeblich nicht eingereicht.


\section{Termine und Fristen}
Die folgenden Termine und Fristen wurden eingehalten:

\begin{itemize}[label=$\Box$]
\sym{0} Festlegung Wahltermin 30 Tage vor Wahl % §10.1
\sym{0} Wahl Wahlleitung und Wahlausschuss 30 Tage vor Wahl % §11.2
\sym{0} Konstituierende Wahlausschusssitzung 25 Tage vor Wahl % §13.4
\sym{0} Festlegung Termine, Fristen und Orte 25 Tage vor Wahl % §13.5
\sym{0} Übernahme Wählendenverzeichnis 19 Tage vor Wahl % §14.1
\sym{0} Wahlbekanntmachung 18 Tage vor Wahl % §15.1
\sym{0} Auslage Wählendenverzeichnis an mindestens 3 Werktagen vor Frist für Einsprüche % §14.3
\sym{0} Frist für Kandidaturen und Anträge 13 Tage vor Wahl bis 10 Tage vor Wahl % §13.5 
\sym{0} Konstituierende FSV-Sitzung 5 bis 14 Tage nach Wahl, oder im Fall einer Wahl-Vollversammlung sofort % §22 bzw. §27.9
\end{itemize}

% Die Auslageorte des Wählendenverzeichnisses wurden nicht explizit festgelegt, sondern lediglich implizit durch Veröffentlichung der Wahlbekanntmachung.

% Die Wahlvorschläge wurden erst A Tage vor Beginn der Wahl bekannt gegeben.

% Protokolle mit Wahl von Wahlleitung und Wahlausschuss sowie der Wahlausschusssitzungen mit entsprechenden Beschlüssen fehlen.

% Die Wahlbekanntmachung wurde 1 Tag zu spät veröffentlicht.

% Die Wahlergebnisbekanntgabe enthält kein Veröffentlichungsdatum.

% Die Frist zur Einreichung von Wahlvorschlägen war mit dem DD.MM.YYYY zwei Tage zu spät angegeben.

% Auf der Bekanntgabe der Wahlvorschläge ist kein Veröffentlichungsdatum angegeben.

% Die konstituierende Sitzung des Wahlausschusses (DD.MM.) fand scheinbar nach der zweiten Wahlausschusssitzung (DD.MM.) statt.

% Die konstituierende Sitzung des Wahlausschusses fand am DD.MM. statt, es existiert ein Protokoll einer FSR-Sitzung vom DD.MM., auf der der Wahlausschuss gewählt wurde. Das ist seltsam.

\section{Wahlausschuss}
\begin{itemize}[label=$\Box$]
	\sym{0} Der Wahlausschuss besteht aus Wahlleitung und mindestens zwei weiteren Mitgliedern % §11.1
	\sym{0} Mitglieder des Wahlausschusses sind keine Kandidierende % §11.1
\end{itemize}

\section{Wahlverfahren}
\begin{itemize}[label=$\Box$]
	\sym{0} Das Wahlverfahren steht im Einklang mit der Fachschaftssatzung % §3.1
	\sym{0} Die Anzahl der zu wählen Personen steht im Einklang mit der FSWO % §5 bei FSV bzw §6 bei FSR. Beachte §6.2 letzter Absatz
	\sym{0} Anträge zum Wahlverfahren lagen nicht vor
\end{itemize}
ODER
\begin{itemize}[label=$\Box$]	
	\sym{0} Anträge zum Wahlverfahren wurden ordnungsgemäß behandelt % §28.2 und §13
\end{itemize}

\section{Kandidaturen}
\begin{itemize}[label=$\Box$]
\sym{0} Kandidierende sind wahlberechtigt und wählbar % §14.1
\sym{0} Kandidaturen sind ordnungsgemäß % §16.1+2
\sym{0} Kandidierende sind weder Wahlausschussmitglieder noch an der Auszählung beteiligte Wahlhelfende % §11.1 und §20.2
\sym{0} Die Liste der zugelassenen Kadidaturen wurde nach § 13 Abs. 4 veröffentlicht
\end{itemize}

% Bei 13 zugelassenen Wahlbewerbungen mit insgesamt 13 Kandidierenden finden die Regelungen zum Wahlverfahren in Sonderfällen keine Anwendung.

% Die Kandidatin \textit{NAME NAME} ist nicht im Wählerverzeichnis zu finden.

% Bei \textit{NAME NAME} fehlt die Unterschrift zur Zustimmung zur Aufnahme in den Wahlvorschlag. Für den WPAF ist der Wille zur Kandidatur damit nicht nachzuvollziehen.

% Bei \textit{NAME NAME} ist der Unterstützer \censor{GEHEIM GEHEIM} nicht im Wählerverzeichnis zu finden. Somit fehlt eine Unterstützungsunterschrift.

% Bei den Listenplätzen A und B fehlt bei den Adressen die Angabe von Postleitzahl und Ort.

% Bei allen Kandidaturen fehlen Anschrift und E-Mail-Adresse. Bei allen Kandidaturen bis auf eine fehlt die Matrikelnummer. 

\section{Wahlunterlagen}
\begin{itemize}[label=$\Box$]
\sym{0} Urnenbuch korrekt geführt % §19.1, §19.7
\begin{itemize}[lable=$\Box$]
    \sym{0} Öffentliche Versiegelung der Urne vor der Wahl
    \sym{0} Über- und Rückgabe der Urne
    \sym{0} Ent- und Versiegelungen der Urne
    \sym{0} Hinzukommen und Verlassen von Wahlhelfenden
    \sym{0} Jeder wählende Person mit laufender Nummer, Name, Matrikelnummer und Unterschrift.
    \sym{0} Öffentliche Entsiegelung der Urne vor der Auszählung
    \sym{0} Unterschriften der beteiligten Wahlausschussmitglieder und Wahlhelfenden
\end{itemize}
\sym{0} Stimmzettel enthalten alle notwendigen Daten und Ankreuzfelder % §17 
\begin{itemize}[label=$\Box$]
    \sym{0} Stimmzettel sind einheitlich gedruckt und geschnitten
    \sym{0} Das zu wählende Organ und die Anzahl an zu vergebener Stimmen sind korrekt benannt
    \sym{0} Die Reihenfolge der Kandidierenden ist zufällig gewählt
    \sym{0} Ein Freifeld nach § 17 Abs. 3 FSWO
\end{itemize}
\end{itemize}

% Der Stimmzettel enthielt ein Freifeld, welches es nach § 17 Abs. 3 FSWO nicht haben dürfte.

% Das Urnenbuch wurde lediglich mit Bleistift abgeschlossen.

% Ein Urnenbuch ist nicht vorhanden.

% Auf dem Stimmzettel ist ein falsches zu wählendes Organ angegeben. 

% Die Zusammenfassung aller eingereichten Kandidaturen zu einer gemeinsamen Liste durch den Wahlausschuss ist nicht statthaft.


\section{Rahmenbedingungen}
\begin{itemize}[label=$\Box$]
\sym{0} Wahlbekanntmachung enthält alle vorgeschriebenen Inhalte nach § 15 Abs 2 FSWO
\sym{0} Korrekte Daten in Wahlbekanntmachung % Siehe "Termine und Fristen"
\sym{0} Stimmzettel wurden korrekt ausgezählt % §20
\sym{0} Wahlergebnis enthält alle vorgeschriebenen Inhalte % §21.1
\sym{0} Wahlergebnis wurde korrekt festgestellt (Sitze, Verfahren) % §$§7.2 ff.
\sym{0} Wahlergebnis wurde auf der Bekanntmachungswebseite der Studierendenschaft veröffentlicht % §21.5
\end{itemize}

% In der Wahlbekanntmachung fehlt der Hinweis auf die Möglichkeit eines Antrags auf Briefwahl, sowie die bei der Briefwahl zu beachtenden Fristen.

% In der Wahlbekanntmachung ist für die Einreichungsfrist der Wahlbewerbungen keine Uhrzeit angegeben. Üblicherweise wird eine Uhrzeit angegeben.

% In der Wahlbekanntmachung enthält die Darstellung des Wahlsystems eine falsche Angabe ([QUOTE])

% Bei der Neuauszählung der Stimmzettel wurde eine vom Wahlausschuss dem Kandidaten \textit{NAME NAME} zugerechnete Stimme vom WPAF als ungültig eingestuft, da neben der Stimmabgabe ein Zusatz enthalten war.

% Im Wahlergebnis fehlen die Zahl der ungültigen Stimmen sowie die Zahl der auf jede Liste entfallenden Stimmen.

% Die Feststellung über die gewählten Personen im Wahlergebnis ist falsch. Nach Anwendung des D'Hondt-Verfahrens entfallen auf die beiden Erstplatzierten Listen je 2 Sitze, auf Positionen 3 -- 9 jeweils 1 Sitz, auf die übrigen Listen 0 Sitze. Die Plätze 10 und 11 wären somit nicht gewählt.

% Die Wahlbekanntmachung enthält nicht explizit einen Hinweis darauf, dass nur wählen kann, wer im Wählerverzeichnis eingetragen ist.

% In der Wahlbekanntmachung sind A zu wählende Mitglieder angegeben. Zu wählen sind jedoch lediglich B Personen.

% In der Bekanntgabe des Wahlergebnisses fehlen die Angabe darüber, welche Kandidierenden gewählt sind und welche nicht. 

% In der Bekanntgabe des Wahlergebnisses fehlt der Hinweis auf die vorgeschriebene Form des Einspruchs gegen das Wahlergebnis und den Wahlprüfungsausschuss der Fachschaftenkonferenz (nicht Wahlausschuss) als zuständige Stelle.

% In der Wahlbekanntmachung fehlen die Bezeichnung des zu wählenden Organs sowie die Zahl der zu wählenden Mitglieder.

% Das für die Entgegennahme der Kandidaturen zuständige Organ ist falsch angegeben.

% Die Darstellung des Wahlsystems nach § 7 FSWO besteht lediglich aus dem Verweis auf § 7 FSWO.

% Die Zahl der zu wählenden Mitglieder ist nicht "`mindestens A"', sondern genau A.

% Die Frist für X ist in der Wahlbekanntmachung falsch angegeben.

% In der Wahlbekanntmachung ist ein falscher Stichtag für die Wahlberechtigung angegeben.

\section{Briefwahl}
\begin{itemize}[label=$\Box$]
\sym{0} Briefwahlanträge lagen nicht vor.
\end{itemize}

 ODER
 
\begin{itemize}[label=$\Box$]
\sym{0} Briefwahlanträge und -Unterlagen sind ordnungsgemäß und vollständig % §18.1 + §18.4
\sym{0} Briefwahlanträge wurden ordnungsgemäß behandelt. % §18.2
\end{itemize}

% Der WPAF geht davon aus, dass keine Briefwahlanträge vorlagen.

\section{Wahlprüfung}
\begin{itemize}[Label=$\Box$]
    \sym{0} Einsprüche gegen die Gültigkeit der Wahl bis 14 Tage nach der Wahl % §25.2
    \sym{0} Veranlassung der Wahlprüfung durch FSK oder WPAF bis 30 Tage nach der Wahl % §25.7 bzw. §25.8
    \sym{0} Anforderung von Dokumenten bis 7 Tage nach Prüfungsanforderung % §25.9
    \sym{0} Übermittlung der Dokumente bis 30 Tage nach Anforderung  % §25.10
    \sym{0} Beschlussempfehlung bis 30 Tage nach Frist für Übermittlung der Dokumente % §25.11
    
\end{itemize}


\section{Weitere Anmerkungen}

% Die Stimmzettel enthielten fälschlicherweise ein Freifeld. Dies hatte jedoch keine Auswirkung auf die Stimmverteilung, da niemand davon Gebrauch gemacht hat.

% Um die fehlerhafte Einreichung von Kandidaturen zu vermeiden, empfiehlt sich die Bereitstellung einer FSWO-konformen Kandidaturvorlage. 

% Das Wahlausschussprotokoll zur Ergebnisfeststellung fehlt. 

% Das Protokoll, in dem gemäß § 12 Abs. 6 die Unversehrtheit der Siegel festgestellt wird, fehlt.

% Die Unterschriften der Wahlhelfenden gemäß § 12 Abs. 3 werden üblicherweise im Urnenbuch geleistet. Sie sind nicht vorhanden.

% Die Entscheidung über die Gültigkeit der Kandidaturen durch den Wahlausschuss (§ 8 Abs. 6) ist in keinem Protokoll festgehalten.

% Die Stimmzettel sind schief geschnitten, wurden auf unterschiedliches Papier gedruckt und sind dadurch möglicherweise unterscheidbar. Es wird dringend empfohlen, die Stimmzettel professionell herstellen, oder mindestens durch geeignetes Gerät schneiden zu lassen.

% Zur Stimmabgabe wurden Stifte in mehreren Farben verwendet.

% Es ist unklar, ob die Wahlleitung oder Mitglieder des Wahlausschusses FSR- oder FSV-Mitglied waren.

% Wenn keine Kandidaturen eingehen, ist gemäß § 26 FSWO einmalig eine Wiederholungswahl anzusetzen. Eine willkürliche Auswahl von Personen, die dann ersatzweise auf den Stimmzettel geschrieben werden, ist \textbf{NICHT} zulässig.

% Die Fachschaftssatzung suggeriert in § 12, dass eine Wahlvollversammlung abgehalten werden soll.

\section{Fazit}

% Das Wahlvehrfahren wurde in der Wahlbekanntmachung fälschlicherweise als Listenwahl bezeichnet aber richtig als Persönlichkeitswahl durchgeführt. In der Aufstellung des Wahlergebnisses ist ebenfalls die rede von einer Listenwahl.

% Die fehlende Unterschrift der Kandidatin \textit{NAME NAME} ist nur dann schwerwiegend, wenn kein Wille zur Kandidatur vorhanden war. Die Kandidatin und ggf. weitere Zeugen sind zu befragen. Falls kein Wille zur Kandidatur vorhanden war, ist die Wahl ohne Berücksichtigung dieser Kandidatin ab dem Zeitpunkt der Zulassung der Wahlbewerbungen zu wiederholen.

% Der fehlende Hinweis auf die Möglichkeit der Briefwahl sowie die zu beachtenden Fristen in der Wahlausschreibung stellt einen erheblichen Mangel der Wahlausschreibung dar. Da bei Fachschaftswahlen in der Regel seltenst Briefwahl beantragt wird und davon auszugehen ist, dass Personen, die ihre Stimme nicht persönlich abgeben konnten, sich im Zweifelsfall bei der Wahlleitung nach Alternativmöglichkeiten erkundigt hätten, stuft der WPAF diesen Mangel jedoch nicht als schwerwiegend ein, sofern nicht Wahlberechtigten explizit die Möglichkeit der Briefwahl verweigert wurde. Hierfür gibt es jedoch keine Indizien.

% Die Aufstellung des Endergebnisses ist fehlerhaft und daher gemäß § 16 Abs. 4 FSWO zu wiederholen.

% Es wurden bei der Durchführung der Wahl keine Mängel festgestellt, aufgrund derer die Wahl ganz oder teilweise für ungültig erklärt werden müsste. Bei der Wahlsicherung werden die fehlenden Unterschriften der Wahlhelfenden gemäß § 12 Abs. 3 FSWO sowie das fehlende Protokoll gemäß § 12 Abs. 6 FSWO bemängelt. Eine Auswirkung auf die Sitzverteilung lässt sich jedoch nicht feststellen.

% Aufgrund der erheblichen Mängel im Wahlverfahren (fehlende Kandidaturen, Wahldurchführung, Besetzung des Wahlaussschusses, \dots) empfiehlt der Wahlprüfungsausschuss dringend, die  gesamte Wahl für ungültig zu erklären (§ 16 Abs. 5 FSWO). Die alte FSV müsste dann unverzüglich einen neuen Wahltermin festlegen und einen Wahlausschuss wählen.

% Das fehlende Protokoll zur Entscheidung über die Gültigkeit der Wahlvorschläge (§ 8 Abs. 6 FSWO) und zur Festlegung des Wahlverfahrens im Sonderfall (§ 9 Abs. 1 FSWO) wird bemängelt. Da offenbar dennoch ein geeignetes Wahlverfahren angewendet wurde und alle Wahlvorschläge, die keine groben Mängel aufweisen, an der Wahl teilnehmen durften, ist dies nicht schwerwiegend.

% Vor dem Hintergrund einer globalen Pandemie wurde auf die Möglichkeit einer Breifwahl nicht ausreichend hingewiesen.

% Trotz umfangreicher Mängel konnte kein Einfluss auf die Sitzeverteilung festgestellt werden. Die Wahl ist somit anzunehmen.

% Bei der Wahl traten geringfügige Mängel auf, die aus unserer Sicht aber keinen Einfluss auf die Stimmverteilung hatten.

% Das Wahlergebnis wurde nicht auf der Bekanntmachungsplattform der Studierendenschaft veröffentlicht (vgl. § 21 Abs. 5)

\section*{Beschlussempfehlung}

Der WPAF empfiehlt der Fachschaftenkonferenz folgenden Beschluss:

\vspace{1em}

% Die Aufstellung des Endergebnisses ist fehlerhaft. Die Aufstellung des Endergebnisses wird daher aufgehoben und eine erneute Feststellung angeordnet (vgl. 25 Abs. 4 FSWO).

% Die Wahl der Fachschaftsvertretung FACHSCHAFT im Zeitraum \wahltermin~ wird für ungültig erklärt. Sie ist gemäß § 26 Abs. 3 FSWO vollständig zu wiederholen.

% Die Wahl der Fachschaftsvertretung X im Zeitraum \wahltermin~ wird teilweise für ungültig erklärt. Sie ist gemäß § 26 Abs.3 teilweise zu wiederholen.

% Die Wahl des Fachschaftsrates \fachschaft~im Zeitraum \wahltermin~wird für gültig erklärt. 

gez. \vorsitz\\
Vorsitz des WPAF
\end{document}
